\section{Bernoulli Bond Percolation}
\subsection{Point critique}
\begin{itemize}[label=\textbullet, left=0pt]
    \item $\theta(p) :=\probP_p[\lVert C \rVert = \infty] $
    \item $\psi(p) :=\probP_p["\text{Il existe un \textit{cluster} infini}"] $
\end{itemize}
\proposition(Existence d'un point critique)\\[10pt]\label{thm:existence}
Il existe $p_c \in [0,1]$ tel que
\begin{enumerate}[label=\roman*)]
    \item $ \psi(p) = 0$ si $p < p_c(d) $\\[0pt]
    \item $ \psi(p) = 1$ si $p > p_c(d) $\\
\end{enumerate}
\begin{proof}
    Soit $X_e \sim \mathcal{U}(0,1)$ et $p \in [0,1]$. On peut facilement voir que
    \begin{align*}
        \probP_p[\mathbb{1}_{X_e<p} = 1]&= \probP_p[x_e<p] = \int_{0}^{p}dx = p\\
        \probP_p[\mathbb{1}_{X_e<p} = 0]&= \probP_p[x_e\geq p]=\int_{p}^{1}dx =  1- p
    \end{align*}
    On a donc que $\mathbb{1}_{X_e<p}\sim \mathcal{B}ern(p)$.
    On prend $p_1,p_2 \in [0,1] $ avec $p_1<p_2$.
    Si $\mathbb{1}_{X_e<p_1} = 1$ alors $\mathbb{1}_{X_e<p_2} =1$. Cela implique que $\mathbb{1}_{X_e<p}$ est croissant sur $p$ et donc que $\omega$ est croissant sur $p$. 
    Si $\omega^{p_1}$ réalise un \textit{cluster} infini, alors $\omega^{p_2}$  le réalise aussi. 
    On peut en déduire que $\psi$ est croissant sur $p$.
    Il est évident que $\psi (0) = 0$ et $\psi(1)=1$. 
    Cela implique qu'il existe un unique $p_c \in [0,1]$ tel que $\psi(p) = 1$ si $p > p_c(d)$ et $\psi(p) = 0$ si $p < p_c(d)$.
\end{proof}
\begin{theorem}[]

    $$
    p_c(d) \in (0,1) \text{ pour $d\geq2$}
    $$
\end{theorem}
\begin{proof}
    On va d'abord montrer que si $\psi(p) = 0$ alors $p_c(d)>0$. Pour cela, on va encore utiliser l'argument de la monotonie. Soit $p_1,p_2 \in (0,1]$, tel que $p_1 \leq p_2$. On a déjà montré \ref{thm:existence} que $\psi$ est croissante sur $p$. Donc si $\psi(p_1) = 0$ alors $\psi(p_2) = 0$.Cela implique que $\psi(\hat{p}) = 0 \forall\hat{p} \in [0,p]$ et donc $p_c>0$.\\

    Si $\psi(p) > 0$ alors $\exists x \in \mathbb{Z}^d$ tel que $\theta(p)>0$.


\end{proof}

