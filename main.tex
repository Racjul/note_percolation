%Package I use

\documentclass[12pt,a4paper]{article}
\usepackage[utf8]{inputenc}
\usepackage[french]{babel}
\usepackage[T1]{fontenc}
\usepackage{amsmath, amssymb, amsthm}
\usepackage{geometry}
\geometry{margin=1in}
\usepackage{graphicx}
\usepackage{fancyhdr}
\usepackage{titlesec}
\usepackage{lmodern} 
\usepackage{enumitem}
\usepackage{csquotes} 
\usepackage{bbold}
\usepackage[backend=biber,style=numeric]{biblatex}
\addbibresource{references.bib} % This loads your .bib file
\titleformat{\section}{\normalfont\Large\bfseries}{\thesection}{1em}{}
\titleformat{\subsection}{\normalfont\Large\bfseries}{\thesubsection}{1em}{}
\titleformat{\subsubsection}{\normalfont\Large\bfseries}{\thesubsubsection}{1em}{}

\usepackage{xcolor}
\usepackage[colorlinks=false, pdfborder={0 0 2}, linkbordercolor={red!50!black}, citebordercolor={blue!50!black}, urlbordercolor={green!50!black}]{hyperref}


\newcommand{\probP}{\text{I\kern-0.15em P}}
\newtheoremstyle{break}
  {\topsep}{\topsep}%
  {\itshape}{}%
  {\bfseries}{}%
  {\newline}{}%
%\theoremstyle{break}
% Définitions pour les environnements théorèmes
\newtheorem{theorem}{Théorème}[section]
\newtheorem{lemma}[theorem]{Lemme}
\newtheorem{proposition}[theorem]{Proposition}
\newtheorem{corollary}[theorem]{Corollaire}
\theoremstyle{definition}
\newtheorem{definition}[theorem]{Définition}


\setlength{\parindent}{20pt}

\begin{document}

\pagestyle{fancy}
\fancyhead[l]{Julien Racette}
\fancyhead[c]{Stage été 2025}
\fancyhead[r]{\today}
\fancyfoot[c]{\thepage}
\renewcommand{\headrulewidth}{0.2pt}
\setlength{\headheight}{15pt} 

\title{Note percolation\\ Stage été 2025}
\author{Julien Racette}
\date{\today}
\makeatletter
\renewcommand{\maketitle}{
    \begin{titlepage}
        \vspace*{\fill} % Vertically center the title block
        \begin{center}
        {\LARGE\bfseries \@title \par} % Title
            \vspace{1.5em}
            {\large \@author \par} % Author
            \vspace{1em}
            {\large \@date \par} % Date
        \end{center}
        \vspace*{\fill} % Push to center
    \end{titlepage}
}
\makeatother
\maketitle
\tableofcontents
\pagebreak

\subsection{Notation et quelques définitions}
J'utilise une notation similaire à celle de  \cite{duminilcopin2022introduction} et \cite{grimmett1999percolation}. On dit que $\mathbb{L}^d = (\mathbb{Z}^d,\mathbb{E}^d)$ est un réseau de dimension $d$ où $\mathbb{Z}^d$ est l'ensemble des sommets et $\mathbb{E}^d$ est l'ensemble des arêtes entre les points. On note $x \sim y$ si $x$ et $y$ sont voisins. On a  $x\leftrightarrow y$ si il existe une arête entre eux. On appelle $\omega=(\omega_e:e\in \mathbb{E}^d) \in \Omega$ la configuration de $\mathbb{L}^d$. On dit que $\omega^1 \leq \omega^2 \text{ si } \omega^1_e \leq \omega^2_e \text{, }\forall e \in \mathbb{E}^d$.
On dit que $\omega_e = 1$ est ouvert et $\omega_e = 0$ est fermé. On appelle \textit{cluster} une composante connexe.

\section{Bernoulli Bond Percolation}
\subsection{Point critique}
\begin{itemize}[label=\textbullet, left=0pt]
    \item $\theta(p) :=\probP_p[\lVert C \rVert = \infty] $
    \item $\psi(p) :=\probP_p["\text{Il existe un \textit{cluster} infini}"] $
\end{itemize}
Par invariance de la translation du réseau et de la mesure de probabilité, on peut dire que la composante connexe étudié est toujours centré en 0. 
\proposition(Existence d'un point critique)\\[10pt]\label{thm:existence}
Il existe $p_c \in [0,1]$ tel que
\begin{enumerate}[label=\roman*)]
    \item $ \psi(p) = 0$ si $p < p_c(d) $\\[0pt]
    \item $ \psi(p) >0$ si $p > p_c(d) $\\
\end{enumerate}
\begin{proof}
    Soit $X_e \sim \mathcal{U}(0,1)$ et $p \in [0,1]$. On peut facilement voir que
    \begin{align*}
        \probP_p[\mathbb{1}_{X_e<p} = 1]&= \probP_p[x_e<p] = \int_{0}^{p}dx = p\\
        \probP_p[\mathbb{1}_{X_e<p} = 0]&= \probP_p[x_e\geq p]=\int_{p}^{1}dx =  1- p
    \end{align*}
    On a donc que $\mathbb{1}_{X_e<p}\sim \mathcal{B}ern(p)$.
    On prend $p_1,p_2 \in [0,1] $ avec $p_1<p_2$.
    Si $\mathbb{1}_{X_e<p_1} = 1$ alors $\mathbb{1}_{X_e<p_2} =1$. Cela implique que $\mathbb{1}_{X_e<p}$ est croissant sur $p$ et donc que $\omega$ est croissant sur $p$. 
    Si $\omega^{p_1}$ réalise un \textit{cluster} infini, alors $\omega^{p_2}$  le réalise aussi. 
    On peut en déduire que $\psi$ est croissant sur $p$.
    Il est évident que $\psi (0) = 0$ et $\psi(1)=1$. 
    Cela implique qu'il existe un unique $p_c \in [0,1]$ tel que $\psi(p) >0$ si $p > p_c(d)$ et $\psi(p) = 0$ si $p < p_c(d)$.
\end{proof}


\textbf{Note}:   
On dénote par $N(p)$ le nombre de cluster infini
On dénote par $N(p)$ le nombre de cluster infini
On pourrait montrer que $\psi(p) =1 \text{ si } p>p_c$ en utilisant la loi du zéro-un de Kolmogorov puisque si on ferme ou ouvre un nombre fini d'arête, cela ne changera pas la nature du résultat et les variables aléatoire sur chaque arête sont iid.
\begin{theorem}[]

    $$
    p_c(d) \in (0,1) \text{ pour $d\geq2$}
    $$
\end{theorem}
\begin{proof}
    On va d'abord montrer que si $\psi(p) = 0$ alors $p_c(d)>0$. Pour cela, on va encore utiliser l'argument de la monotonie. Soit $p_1,p_2 \in (0,1]$, tel que $p_1 \leq p_2$. On a déjà montré \ref{thm:existence} que $\psi$ est croissante sur $p$. Donc si $\psi(p_1) = 0$ alors $\psi(p_2) = 0$.Cela implique que $\psi(\hat{p}) = 0 \forall\hat{p} \in [0,p]$ et donc $p_c>0$.\\
    Si $\psi(p) > 0$ alors $\exists x \in \mathbb{Z}^d$ tel que $\theta(p)>0$.
\end{proof}
\subsection{Unicité du cluster infini}
\begin{theorem}On dénote par $N(p)$ le nombre de clusters infinis dans $\mathbb{L}^d$.\\
    \begin{equation}
        N(p) \in \{0,1,\infty\}
    \end{equation}
\end{theorem}
\begin{proof}
    On défini les évènements suivants:$\mathcal{E}_{\leq1}$ =\{"Il existe 0 ou 1 cluster infini"\}, $\mathcal{E}_{\infty}$= \{"Il existe une infinité de cluster infini"\} et $\mathcal{E}_{\leq \infty}$ = \{"Il existe un nombre fini clusters infinis"\}.\\ On examine le cas $p>p_c$, car le reste est trivial. On veut montrer que si le nombre de cluster est fini, alors il est égal à 0 ou 1.
    Autrement dit, $\probP[\mathcal{E}_{\leq \infty}] = 1 \implies \probP[\mathcal{E}_{\leq1}] = 1 $. On prend $\Lambda_n:=[-n,n]^d$ un carré de longueur $2n$ et l'évènement:$$
    \mathcal{F}=\{\text{ "Tous les clusters infinis intersectent $\Lambda_n$" \}}.
    $$  On prend maintenant l'évènement $E_n=\{\text{"Tous les arêtes dans $\Lambda_n$ sont fermées"\}}$. Or, si les arêtes de $\Lambda_n$ sont fermées et que tous les clusters infinis intersectent $\Lambda_n$, alors il existe 1 seul cluster infini.
    On a donc que 
    \begin{align*}
        \probP_p[\mathcal{E}_\leq1] &\geq \probP_p[\mathcal{F} \cap E_n]\\
                                    &=\probP_p[\mathcal{F}]\cdot\probP_p[E_n]\tag{Indépendance}\\ 
                                    &\geq \probP_p[\mathcal{F}] \cdot p^{\lVert \mathbb{E}_n \rVert}\\ 
                                    &\geq c \text{ }\probP[\mathcal{E}_{<\infty}]\cdot p^{\lVert \mathbb{E}_n \rVert}\tag{$n$ assez grand}\\
                                    &>0
    \end{align*}
    %TODO
    Or, puisque $\probP_p[\mathcal{E}_{\leq1}] > 0$ alors $\probP_p[\mathcal{E}_{\leq1}] = 1$ par l'argument d'ergodicité. 
    \begin{align*}
        \probP_p[\mathcal{E_{<\infty}}] &= \probP_p\left[\bigcup_{i\geq1}\mathcal{E}_{i}\right]\\
                                        &= 1 + \probP_p\left[\bigcup_{i\geq2}\mathcal{E}_{i}\right]\\
                                        &\implies \probP_p\left[\bigcup_{i\geq2}\mathcal{E}_{i}\right] = 0
    \end{align*}
    Et donc la probabilité d'avoir un nombre fini $\geq2$ de clusters infinis est nulle.\\
\end{proof}
\begin{theorem}[Unicité du cluster infini]
    Si $\theta(p) =1 $, alors
    $$
    \probP_p[\text{"Il existe un unique cluster infini" }] = 1
    $$
    
\end{theorem}
\begin{proof}    
\end{proof}

\nocite{heckel2010percolation}
\printbibliography
\end{document}

